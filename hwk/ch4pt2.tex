\documentclass[nobib]{tufte-handout}
\usepackage{amsmath}
\usepackage{setspace}
\usepackage{hyperref}
\usepackage{booktabs}


\setlength{\textheight}{9in} \setlength{\topmargin}{-.5in}
\setlength{\textwidth}{6.5in} \setlength{\oddsidemargin}{0in}
\setlength{\evensidemargin}{0in}

\title{COMP 235 --- Homework --- Ch 4.6 -- }
\author{  }
\date{ }
 

\begin{document}
\maketitle


\begin{enumerate}
    \item Consider the following bytes: $x = 0xE4$, $y = 0x6B$, $z=0xF1$.  Compute the following. Show work. 
    \begin{enumerate}
        \item $\sim$y
        \vspace{1.5in}
        \item x \& z
        \vspace{1.5in}
        \item y \^{} x
        \vspace{1.5in} 
        \item z | y
        \vspace{1.5in}
        \item $\sim$(y \& z)
        \newpage \thispagestyle{empty}
        \item z << 4
        \vspace{1.5in}
        \item y >> 3 (arithmetic)
        \vspace{1.5in}
        \item z >> 4 (arithmetic)          
        \vspace{1.5in}
        \item z >> 4 (logical)         
        \vspace{1.5in}
        \item (y << 7) >> 7 (arithmetic)
    \end{enumerate}
    \newpage
    \thispagestyle{empty}

    \item Convert the following binary values to decimal.
    \begin{enumerate}
        \item 1011.011010
        \vspace{2in}
        \item 0.1101
        \vspace{2in}
        \item 10.001001001
        \newpage \thispagestyle{empty}
    \end{enumerate}
    \item Convert the following binary numbers to normalized (leading 1) scientific notation. 
    \begin{enumerate}
        \item 1011.001 
        \vspace{2in}
        \item 0.001001
        \vspace{2in}
        \item 11.010
        \vspace{2in}
        \item 0.011111
        \newpage \thispagestyle{empty}        
    \end{enumerate}
    \item For simplicity sake, let's consider a 17 bit floating point notation that is the same as what we get in the book but with only 1 byte for the significand. So that is 1 bit for the sign, 1 byte for the exponent (with a bias of 127), and 1 byte for the significand. For each of the following bit strings, list the binary number it represents in scientific notation and the value in decimal. 
    \begin{enumerate}
        \item 1 00010110 00001110
        \vspace{2in}
        \item 0 11010011 11011111 
        \vspace{2in} 
    \end{enumerate} 
    \item Give the binary string to represent the following binary values using the same 17 bit floating point number representation listed above.
    \begin{enumerate}
        \item $110.111$
        \vspace{2in}
        \item $-1.011 \times 2^{27}$
        \newpage \thispagestyle{empty}
    \end{enumerate}
    \item (Bonus!?) Show how to represent these decimal values using the 17 bit floating point representation from the previous problem. 
    \begin{enumerate}
        \item -13.6875
        \vspace{2in}
        \item 0.087890625
    \end{enumerate}

\end{enumerate}

\end{document}