\documentclass[nobib]{tufte-handout}
\usepackage{amsmath}
\usepackage{setspace}
\usepackage{hyperref}
\usepackage{booktabs}


\setlength{\textheight}{9in} \setlength{\topmargin}{-.5in}
\setlength{\textwidth}{6.5in} \setlength{\oddsidemargin}{0in}
\setlength{\evensidemargin}{0in}

\title{COMP 235 --- Assembly Quizlet}
\author{  }
\date{ }
 

\begin{document}
\maketitle

You've been given a short C program along with the assembly generated from that file. Analyze the assembly and determine the answers to the following:

\begin{enumerate}
    \item For \textit{main}
    \begin{enumerate}
        \item attribute all lines of assembly to line of C or to function call prologue and epilogue code done at the assembly level. Label assembly line numbers with C line numbers.  
        \item determine the size of the stack frame in bytes
        \item determine the location within the frame for each of the 3 arrays. List the locations relative to \%rbp. For example, $\left[-10,-5\right]$ would refer to bytes \%rbp-10 through \%rbp-5. 
        \item draw a diagram showing how each byte of the stack frame is used. If bytes are allocated on the stack but not directly used, label them as unused. 
        \item determine the primary operand specifier for the variables A, B, and C. 
    \end{enumerate}
    \item For \textit{initAll}
    \begin{enumerate}
        \item attribute all lines of assembly to line of C or to function call prologue and epilogue code done at the assembly level. Label assembly line numbers with C line numbers. 
        \item determine how much stack space is used
        \item determine the primary operand specifier for variables a,b,c,i, and j. 
    \end{enumerate}
    \item For \textit{matmul}
    \begin{enumerate}
        \item attribute all lines of assembly to line of C or to function call prologue and epilogue code done at the assembly level. Label assembly line numbers with C line numbers. 
        \item determine how much stack space is used
        \item determine the primary operand specifier for variables a,b,c,i,j, and k
    \end{enumerate}
    
\end{enumerate}

We'll spend some time in class getting started on this. This exercise is open book, open notes, open professor, open classmates, but not open internet. Do not consult with AI. 

\end{document}