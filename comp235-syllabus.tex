\documentclass[10pt]{article}
\usepackage{amsmath}
\usepackage{setspace}
\usepackage{hyperref}
\usepackage{booktabs}

\setlength{\textheight}{9in} \setlength{\topmargin}{-.5in}
\setlength{\textwidth}{6.5in} \setlength{\oddsidemargin}{0in}
\setlength{\evensidemargin}{0in}

\title{Syllabus \\ COMP 235 \\ Introduction to Systems Programming \\ \textit{Remote Learning Edition}}
\author{  }
\date{Fall 2020}

\begin{document}
\maketitle

\section{Remote Learning Expectations}

Due to the ongoing COVID-19 pandemic, this course will be delivered remotely. The plan
is to retain most of the regular meetings and opporotunities for help and direct instrution
but to carry out these activities via online tools. Here's what that means, what you can expect,
and what will be expected of you for this mode of delivery.

\begin{itemize}
  \item Remote contant information for the instructor and Zoom meeting links will be posted to the
  course Moodle.

  \item Assignments will be posted and submitted via Moodle or a similar online platform.

  \item Class and Lab sessions will meet at the usual time but will be carried out
  via Zoom. \textit{Attendance expectations for these sessions are the same as
  if the class were meeting in person.}

  \item Prior to class meetings, students will be given reading assignments and possibly
  pre-recorded videos to watch. A short reading quiz and homework problems will often
  accompany the assignment as well. These assignments are expected to be done prior to
  class and will be the way you can expect to encounter and explore new ideas in the course.
  Class meetings and lab sessions will be used to clarify these ideas and test our
  shared understanding of these ideas through additional problems. Class is being
  organized this way in part to make it easier for individuals or the class as a whole
  to transition to a more asynchronous setting should the need arrise.

  \item Expect most class sessions to be recorded for the benefit of students that
  are unable to attend class due to an excused absense or illness. If you do not wish
  to be recorded, please contant the instructor ASAP to work something out.

  \item You are not required to use video and audio during the class meetings, but it is,
  however, encouraged. The chat channel will be monitored and used during class in order
  to facilitate those without audio. If you choose to use audio, please try to keep
  microphones muted unless you are speaking. If you choose to use video, please ensure
  your space and background do not contain inapropriate nor offensive material. (Just
  assume one of my kids is going to see it.)

  \item Office hours will be carried out via Zoom. Whenever possible, give the
  instructor a heads up if you need or want to meet. If you log into the Zoom meeting
  and the session isn't running or nobody is there, then email or txt and the instructor
  will hop right on.

\end{itemize}

If you have any concerns about getting access to the technology you need to keep up with
remote meetings or generally have questions about class exceptions for this semseter,
then please do not hesitate to ask. The goal here is flexiblity and we try to make
things work for whatever you situation happens to be.


\section{Logistics}
\begin{itemize}
\item \textbf{Where: } Center for Science and Business (CSB), Room 309\footnotemark[1]
\item \textbf{When: } MWF 1--1:50pm
\item \textbf{Instructor: } Logan Mayfield
\begin{itemize}
\item \textit{Office: } Center for Science and Business (CSB), Room 344
\item \textit{Phone: } 309-457-2200 % chktex 8
\item \textit{Website: } \url{http://jlmayfield.github.io/}
\item \textit{Email: } lmayfield \textit{at} monmouthcollege \textit{dot} edu
\item \textit{Office Hours: }  By appointment.
\end{itemize}
%\item \textbf{Website: } \url{http://jlmayfield.github.io/teaching/COMP235/}
\item \textbf{Credits: } 1 course credit
\end{itemize}
\emph{Note: This Syllabus is subject to change based on specific class needs. Significant deviations from the syllabus will be discussed in class.}

\footnotetext[1]{This room will not be used for meetings but will be reserved for this
class and contains computers should you need or want them.}

\section{Description, Content, and Learning Goals}

An introduction to low-level programming and computer hardware
organization from a software perspective, emphasizing how application
programmers can use knowledge of the entire system to write better
programs. Introduces C and assembly language. Core topics include data
representation, machine language, the memory hierarchy, and virtual
memory. Further potential topics include processor architecture, code
optimization, and concurrency.

The goal of this course is to produce programmers who understand how
the correctness and performance of their applications are impacted by
the compilation system, operating system, and hardware. After
dedicating some time to learning C, the course aims to cover the
majority of chapters 1-9 of the text.

\begin{center}
\begin{tabular}{ll}
\(\bullet\) Basics of C & \(\bullet\) Memory Hierarchy\\
\(\bullet\) Tour of Systems & \(\bullet\) Linking\\
\(\bullet\) Data Representations & \(\bullet\) Exceptional Control Flow (as time allows)\\
\(\bullet\) Machine Language & \(\bullet\) Virtual Memory\\
\(\bullet\) Code Optimization & \(\bullet\) Processor Architecture (as time allows)\\
\(\bullet\) x86 Assembly Language & \(\bullet\) Dynamic Memory\\
\end{tabular}
\end{center}

\subsection{Textbook}

\begin{quote}
Bryant, Randal E. and O'Hallaron, Davir R.
\textit{Computer Systems: A Programmer's Perspective}. 3rd Edition. Pearson. 2016. ISBN-13: 978-0134092669.
\end{quote}


\subsection{Programming Environment}

This course involves programming in both C and assembly language. All
necessary tools will be available on the department server. All
software for this course is available free of charge and can be found
on the web if students wish to install it on their personal
machines. The instructor cannot guarantee support for installing and
using other development environments.

\section{Workload}
% number of/details on midterms, finals, project, homeworks, quizes, etc

The course workload is as follows:
\begin{center}
  \begin{tabular}{ll}
    \underline{Category} & \underline{Number of Assignments} \\
    Exams & 7 \\
    Programs & 5 \\
    Homework & 8
  \end{tabular}
\end{center}

\subsection*{Exams}

All exams are weighted equally. There is no midterm or final exam in the sense that the exams are worth more than other exams or that they will necessarily take longer than other exams.  Exams will generally focus on material covered since the previous exam but will be in some sense cumulative due to the nature of programming. \textit{Exams during this semseter will be administered via Moodle and will have fixed time windows during which they can be completed.}

\subsection*{Programs}

Several of the chapters will have programming assignments to accompany them. These will be assigned when we begin the chapter and you should start looking at them right away and use them as a motivator for the chapter itself. They will typically be due at the completion of the chapter but before the chapter exam. Programs will be individual efforts.

\subsection*{Homework}

Students will be assigned a set of problems from each chapter of the book covered in the course. These problems are meant to guide reading, prepare the student for in class problems, and survey the material covered by the exam. \textit{While we are engaging in remote learning, homework problems will be given submitted via Moodle. Some problems will be administered as Moodle Quizzes where others will require submitting written code.}


\subsection{Course Engagement Expectations}

The weekly workload for this course will vary by student but on average should be about 13 hours per week.  The follow tables provides a rough estimate of the distribution of this time over different course components.
\begin{center}
\begin{tabular}{ll}
\underline{Assignment Type} & \underline{Time/week} \\
Lectures+Labs       & 3 hours/week \\
Homework          & 3 hours/week \\
Exam Study Time    & 1 hours/week \\
Programs          & 3 hours/week \\
Reading &  2 hours/week \\
\bottomrule
 & 12 hours/week
\end{tabular}
\end{center}


\section{Grades}

This course uses a standard grading scale where percentage grades translate to letter grades as follows:

\begin{center}
\begin{small}
\begin{tabular}{lcl}
\underline{Score} & & \underline{Grade} \\
94--100 & & A \\
90--93 & & A- \\
88--89 & & B+ \\
82--87 & & B \\
80--81 & & B- \\
78--79 & & C+ \\
72--77 & & C \\
70--71 & & C- \\
68--69 & & D+ \\
62--67 & & D \\
60--61 & & D- \\
0--59 & & F
\end{tabular}
\end{small}
\end{center}


Students are always welcome to challenge a grade that they feel is unfair or calculated incorrectly.  Mistakes made in the student's favor will never be corrected to lower a grade.  Mistakes not in the student's favor favor will be corrected.  \textit{Basically, after the initial grading, a score can only go up as the result of a challenge.}



\subsection{Grade Weights}

The final grade is based on a weighted average of particular assignment categories.  Students should be able to estimate your current grade based on your scores and these weights, but you may always visit the instructor \textit{outside of class time} to discuss your current standing and check on some or all of the current course grade.

\begin{center}
  \begin{tabular}{ll}
  \underline{Category} & \underline{Weight} \\
    Exams & 35\% \\ %5 each
    Programs & 30\% \\ %6 each
    Homework & 20\% \\ %2.5 each
    Participation & 15\%
  \end{tabular}
\end{center}

\subsection*{Participation \& Attendance}

Participation grades will be determined based largely on the completion of reading quizzes that come along
with the daily reading and video assignments. Class attendance will be monitored via Zoom. Repeated unexcused absenses will have a negative effect on your participation grade. Whenever possible, let the instructor know of the absence before it occurs. When unexcused absences do occur, it is the student's responsibility to make up for the lost class time and to seek the permission of the instructor to hand-in or complete assignments that are late due to an unexcused absence.

\subsection*{Late Work}

In general, assignments are due at the specified time and no late assignments will be accepted unless an extension was requested prior to the due date. There are, of course, exceptions to this rule and students needing extra time can always contact the instructor for an extension. Do not just give up and eat a zero for the assignment. Ever. There is no penalty in asking for an extension nor is there a limit on extensions.  That being said, there is no guarantee an extension will be given without legitimate need.
\subsection{Calendar}

\textit{This calendar is subject to change based on the circumstances of the course.} A detailed, day-by-day calendar of reading requirements and expected exam dates can be found on the course website.

\begin{center}
\begin{tabular}{llll}
\underline{Week} & \underline{Dates} & \underline{Assignments Due} & \underline{Chapter(s)}\\
1 & 8/18--8/21  &  & Ch. 1. Linux \& C \\
2 & 8/24--8/28 &  &  Linux \& C \\
3 & 8/31--9/4 &  &  Ch 2 \\
4 & 9/7--9/11  & Program 1 &  Ch 2 \\
5 & 9/14--9/18 & Exam 1 & Ch 3 \\
6 & 9/21--9/25 & Program 2 & Ch 3 \\
7 & 9/28--10/2 & Exam 2  &  Ch 5 \\
8 & 10/5--10/9 & & Program 3 Ch 5 \\
9 & 10/12--10/16 & Exam 3 &  Ch 6 \\
10 & 10/19--10/23  & Program 4 & Ch 6 \\
11 & 10/26--10/30 & Exam 4 &  Ch 7 \\
12 & 11/2--11/6 & Exam 5 & Ch 9 \\
13 & 11/9--11/13 &  Program 5. & Ch 9 \\
14 & 11/16--11/20 & Exam 6. & Ch 8 \\
15 & 11/23--11/24 &  THANKSGIVING (W-F)  &  Ch 8 \\
16 & 11/30--12/4 & Exam 7 &  \\
\end{tabular}
\end{center}

\end{document}
