\documentclass[10pt]{article}
\usepackage{amsmath}
\usepackage{setspace}
\usepackage{hyperref}
\usepackage{booktabs}

\setlength{\textheight}{9in} \setlength{\topmargin}{-.5in}
\setlength{\textwidth}{6.5in} \setlength{\oddsidemargin}{0in}
\setlength{\evensidemargin}{0in}

\title{Syllabus \\ COMP 235 \\ Introduction to Systems Programming}
\author{  }
\date{Fall 2024}

\begin{document}
\maketitle

\section{Logistics}
\begin{itemize}
\item \textbf{Where: } Center for Science and Business (CSB), Room 309
\item \textbf{When: } MWF 1--1:50pm
\item \textbf{Instructor: } Logan Mayfield
\begin{itemize}
\item \textit{Office: } Center for Science and Business (CSB), Room 344
\item \textit{Phone: } 309-457-2200 % chktex 8
\item \textit{Website: } \url{http://jlmayfield.github.io/}
\item \textit{Email: } lmayfield \textit{at} monmouthcollege \textit{dot} edu
\item \textit{Office Hours: }  M 9-10am. Tu, 10:30am - 11:30am. W, 2-3pm. Th, 10:30-11:30am. F, 2-3pm. By appointment.
\end{itemize}
\item \textbf{Website: } \url{http://jlmayfield.github.io/teaching/COMP235/}
\item \textbf{Credits: } 1 course credit
\end{itemize}
\emph{Note: This Syllabus is subject to change based on specific class needs. Significant deviations from the syllabus will be discussed in class.}


\section{Description, Content, and Learning Goals}

An introduction to low-level programming and computer hardware
organization from a software perspective, emphasizing how application
programmers can use knowledge of the entire system to write better
programs. Introduces C and assembly language. Core topics include data
representation, machine language, the memory hierarchy, and virtual
memory. Further potential topics include processor architecture, code
optimization, and concurrency.

The goal of this course is to produce programmers who understand how
the correctness and performance of their applications are impacted by
the compilation system, operating system, and hardware. The course will
cover the following chapters of the text:

\begin{center}
\begin{tabular}{ll}
\(\bullet\) Intro to C (ch 1) & \(\bullet\) Basics of the CLI (App 2) \\
\(\bullet\) C in Depth (ch 2) & \(\bullet\) C Debugging (ch 3) \\
\(\bullet\) Data Representations (ch 4) & Assembly Programming (ch 6,7, \& 10) \\
\(\bullet\) Computer Architecture (ch 5) & \(\bullet\) The Memory Hierarchy (ch 11) \\
\(\bullet\) The Operating System (ch 13) & \(\bullet\) Virtual and Dynamic Memory (ch 14) \\
\(\bullet\) \(\bullet\) Parallelization (ch 15) & \\
\end{tabular}
\end{center}

\subsection{Textbook}

\begin{quote}
Matthews, Suzanne J, Newhall, Tia, and Webb, Kevin C.
\textit{Dive Into Systems}. Version 1.2. CC BY-NC-ND 4.0.
\url{https://diveintosystems.org}
\end{quote}


\subsection{Programming Environment}

This course involves programming in both C and assembly language. By and large, all
necessary tools will be available on the department server, but students will be encouraged to work via \textit{SSH} using \textit{VS Code} on their personal machine.  If the server is unavailable and the student does not have access to a suitable development environment, then one can be provided by the department. All software for this course is available free of charge and can be found on the web if students wish to install it on their personal machines. The instructor cannot guarantee support for installing and using other development environments.

\begin{itemize}
  \item VS Code : \url{https://code.visualstudio.com/}
  \item Instructions for SSH+VS-Code Integration : \url{https://code.visualstudio.com/docs/remote/ssh-tutorial}
\end{itemize}

\section{Workload}
% number of/details on midterms, finals, project, homeworks, quizes, etc

The course workload is as follows:
\begin{center}
  \begin{tabular}{ll}
    \underline{Category} & \underline{Number of Assignments} \\
    Exams & 5-7 \\
    Project & 1 \\
    Homework & 4-6 \\
    Portfolio Review \& Self Evaluation Meetings & 4-5
  \end{tabular}
\end{center}

\subsection*{Exams}

We'll wrap up most chapters with an exam. Most will be in-class exams. Some will be open-book. Some might even be take home.

\subsection*{Project}

When learning the C programming language, we'll do a programming project in lieu of an exam. The program will be an individual effort and the majority of the work on the project will be carried out outside of class. 

\subsection*{Homework}

Students will be assigned a set of problems from each chapter of the book covered in the course. These problems are meant to guide reading, prepare the student for in class problems, and survey the material covered by the exam.

\subsection*{Portfolio Review \& Self-Evaluation}

Self-reflection and self-evaluation is a critical component of learning and vital to a growth mindset.
We will keep a portfolio of the work you do throughout the semester. Much of this will be done automatically
by our assignment management and version control software. At regular intervals throughout the semester you will meet, one-on-one, with me to \textit{present your porfolio}, review items from your portfolio that best 
gauge how well you're doing at meeting the course goals and expectations, and discuss how that success maps to 
a letter grade. For more details about the process visit \url{https://jlmayfield.github.io/teaching/ungrading/howto-portfolio}.


\subsection{Course Engagement Expectations}

The weekly workload for this course will vary by student but on average should be about 10.5 hours per week.  The follow tables provides a rough estimate of the distribution of this time over different course components.
\begin{center}
\begin{tabular}{ll}
\underline{Assignment Type} & \underline{Time/week} \\
Lecture/Class      & 3 hours/week \\
Homework          & 3.5 hours/week \\
Exam Study Time    & 1 hours/week \\
Programs          & .5 hours/week \\
Reading &  2.5 hours/week \\
\bottomrule
 & 10.5 hours/week
\end{tabular}
\end{center}


\section{Ungrading \& Final Grades}

This class is largely ungraded. That means your assignments will not be graded for points and your final grade
is not determined by a point-based, numerical grading system. You will get feedback on your work but you will
see points on nothing. You don't earn points for doing work or getting something correct nor do you lose points
for getting something wrong. We're here to learn. Doing the work is how we do that and getting things wrong
some or most of the time is part of learning. For more details visit \url{https://jlmayfield.github.io/teaching/ungrading/howto-portfolio}.

\subsection{Self-Evaluation \& Final Course Grades}

Throughout the semester you'll be asked to engage in regular self-evaluation. This process is described in
detail in additional documentation. Part of the process includes you self-assigning a course grade based on
your self-evaluation. Your self-evaluation and self-assigned grade are then discussed with me in a one-on-one
meeting during which we'll agree upon your current grade. The key here is that \textit{your self-evaluation
and self-assigned grade begins the conversation, not my assigned points.}

Below are some general rules of thumb we'll try to stick to when talking about grades. They relate grades to
course competency expectations and Monmouth College policy.
\begin{itemize}
  \item \textbf{A} - Exceeding course expectations.
  \item \textbf{B} - Meeting and occasionally exceeding course expectations.
  \item \textbf{C} - Meeting course expectations. \textit{This is the minimum grade required to continue on to COMP152. So, a C means you can be successful in a class that builds upon the things learned in this class.}
  \item \textbf{C-} - Mostly meeting course expectations. \textit{This is the minium grade that counts towards a major.}
  \item \textbf{D} - Occasionally meeting course expectations, but mostly not. \textit{Grades in the D range earn credit towards graduation but fall below GPA requirements.}
  \item \textbf{F} - Did not meet course expectations.
\end{itemize}

My hope is that the self-evaluation and self-directed grading process provides a lot of flexibility in terms
of how you can achieve success in this course and meet your grade goals. If you ever have questions or concerns
about self-evaluations and grades, then I'm more more than willing to discuss them with you at any time.

\subsubsection{Participation, Attendance, \& Timely Work}

I do not have strict attendance and deadline policies, per se, but I do have clear expectations. These
expectations are baked into the dispositional attribute of the course competencies. This attribute
includes things like being \textit{professional, responsible, responsive, and self-directed.}

As far as I'm concerned, signing up for this class means you agree to coming to class and lab,
being on time for class and lab, doing assigned work and submitting it on time, and generally participating
in all the class has to offer.  That being said, life happens and people have different priorities.
You might need to miss class or extend a deadline.  So long as you communicate with me about it, as a
professional would with a co-worker, then we won't have a problem. If you simply skip class without
warning, always show up late, or regularly fail to do assigned work in a timely manner, then I expect that
those failures to meet dispositional expectations to be reflected in your self-evaluation.

There is one exception to my ``no grade-based policy'' on assignments and deadlines and that is the
self-evaluations and reflections. The self-evaluation process is critical to this class and in no way
optional. \textbf{If you fail attend the portfolio review meetings or always show up completely un-prepared
then I reserve to give you a final grade of D or lower for the course.} You'll find I can be pretty relaxed
about a lot of other assignments and deadlines, but I draw the line at the self-evaluation process.

\subsubsection*{Academic Honesty}

You don't learn by trying to pass off someone's work as your own. In an ungraded class it makes even less sense to cheat and steal work from somewhere else.  There are no points, you gain nothing from it and you certainly will learn nothing from it. In this ungraded class, academic dishonesty is still not tolerated.

From the Monmouth College Academic Honesty Policy:
\begin{quote}
  ``We view academic dishonesty as a threat to the integrity and intellectual mission of our institution. Any breach of the academic honesty policy - either intentionally or unintentionally - will be taken seriously and may result not only in failure in the course, but in suspension or expulsion from the college. It is each student’s responsibility to read, understand and comply with the general academic honesty policy at Monmouth College, as defined here in the Scots Guide, and to the specific guidelines for each course, as elaborated on the professor’s syllabus.''

  ``The following areas are examples of violations of the academic honesty policy:
  \begin{enumerate}
  \item Cheating on tests, labs, etc;
  \item Plagiarism, i.e., using the words, ideas, writing, or work of another without giving appropriate credit;
  \item Improper collaboration between students, i.e., not doing one’s own work on outside assignments specified as group projects by the instructor;
  \item Submitting work previously submitted in another course, without previous authorization by the instructor.''
  \end{enumerate}

  ``Please note that this list is not intended to be exhaustive.''
\end{quote}

In this course, any violation of the academic honesty policy will have varying consequences depending on the severity of the infraction as judged by the instructor.  Expect violations to be reported to the appropriate Dean and to weaken your case for higher grades at the end of the course. Severe violations can result in an F for the course and expulsion from the course. Do your own work. If you even think something you're doing could be construed as academically dishonest, then ask for guidance and clarification first.


\subsection*{Generative AI Policy}

In general, you are not allowed to use generative AI to generate any portion of your work.  No ChatGPT, no Github CoPilot, none of that.  If you're turning the work into me with your name on it, then it should be your work. Writing or tweaking a prompt to generate a solution is not the same as actually generating the solution on your own. We may explore the use of these tools in developing software systems and writing code, but in such cases you will be explicitly told that you can use AI. If you use AI to generate work without the express permission of the instructor, then you will have committed an act of academic dishonesty and will face appropriate consequences (see above).

While you cannot use AI to generate work, you can use AI as a study guide and a tool to assist in learning. This includes getting new study problems from an AI, having AI summarize or paraphrase portions of a reading when studying (not as part of an assignment!), and otherwise finding creative uses for AI.  If, at any point, you are in doubt about whether or not your use of AI is appropriate for the course, ask.  In fact, I'd love to hear about creative ways you're using AI.  


\section{Academic Support \& Accessibility}

\subsection*{Support Services}
The Academic Support and Accessibility Services Office offers free resources to assist Monmouth College students with their academic success. Programs include Supplemental Instruction for difficult classes, Drop-In and appointment tutoring, and individual Academic Coaching. Our office is here to help all students excel academically, since every student can work toward better grades, practice stronger study skills, and manage their time better. Please email academicsupport@monmouthcollege.edu for assistance.

\subsection*{Accessibility Services}
If you have a disability and/or medical/mental health condition or had academic accommodations in high school or another college, you may be eligible for academic accommodations at Monmouth College under the Americans with Disabilities Act (ADA). Monmouth College is committed to equal educational access. To discuss any of the services offered, please call or meet with Jennifer Sanberg, Associate Director of Academic Support and Accessibility Services. The ASAS office is located on the first floor of the Hewes Library, opposite Einstein’s Bros Bagel. They can be reached at 309-457-2257 or via email at: academicsupport@monmouthcollege.edu

\subsection{Calendar}

\textit{This calendar is subject to change based on the circumstances of the course.} A detailed, day-by-day calendar of reading requirements and expected exam dates can be found on the course website.

\begin{center}
\begin{tabular}{lllll}
\underline{Week} & \underline{Dates} & \underline{Major Assignments} & \underline{Chapter(s)} & \underline{Notes} \\
1 & 8/21--8/23  &  & Ch 4 & \\
2 & 8/26--8/30 &  & Ch 4 & \\
3 & 9/2--9/6 & Ch 4 Hwk & Ch 4, Ch 5 & NO CLASS M. - Labor Day. \\
4 & 9/9--9/13  & Ch 4 Exam & Ch 5 &  \\
5 & 9/16--9/20 & Ch 5 Hwk & Ch 5, Ch 1 & \\
6 & 9/23--9/27 & Ch 5 Exam &  & Ch 1\\
7 & 9/30--10/4 & Ch 1 Hwk & & Ch 1 \\
8 & 10/7--10/9 & Ch 1 Mini-Exam & Ch 1, Ch 2 & FALL BREAK Th-F  \\
9 & 10/14--10/18 &  &  & Ch 2\\
10 & 10/21--10/25 &  &  & Ch 3, Ch 6\\
11 & 10/28--11/1 & C Project & Ch 7 &  \\
12 & 11/4--11/8 & Ch 7 Hwk & Ch 7, Ch 10 & \\
13 & 11/11--11/15 & Ch 7 Exam &  & Ch 11 \\
14 & 11/18--11/22 & Ch 11 Hwk & & Ch 11\\
15 & 11/25--11/26 & & Ch 11 & THANKSGIVING W-F \\
16 & 12/2--12/6 & Ch 11 Exam & & NO CLASS F. - Reading Day Th.\\
17 & 12/7 & Final Exam. Sat. 12/7 6:30 -- 9:30 pm & &  \\
\end{tabular}
\end{center}

\end{document}
